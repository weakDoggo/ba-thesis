% !TEX root = ../main.tex
%
\chapter{Conclusion}
\label{sec:conclusion}

Our study examined the impact of various regularization techniques, particularly \ac{gdc}, on graph-level prediction tasks across two different network types. Regularization techniques are commonly employed in node-prediction tasks across diverse networks to mitigate over-smoothing and overfitting issues.
These methods perturb the values and introduce randomness, leading to improved results.
In graph-level tasks, the final output is a readout of aggregated nodes, making over-smoothing less problematic as the focus shifts from distinguishing individual nodes to capturing collective behavior.
Consequently, introducing controlled randomness might seem unnecessary for such predictive tasks.
This shift in focus from individual nodes to aggregated outcomes challenges the conventional wisdom of applying regularization methods in graph-level prediction tasks.
Despite this counterintuitive aspect, our study aimed to empirically validate the effectiveness of regularization techniques in graph-level prediction tasks.
Our exploration of this uncharted territory was motivated by a desire to challenge existing assumptions and gain a nuanced understanding of the relationship between regularization methods and graph-level predictions.
This unconventional approach enabled a rigorous assessment of current methodologies, providing valuable insights into graph-based machine learning.

% Here come explicit answers to all the research questions
We wanted to determine if \ac{gdc} effectively solves over-fitting and over-smoothing for graph-level prediction tasks.
Specifically, we tried to verify or refute whether any of the four regularization techniques \ac{do}, \ac{ns}, \ac{de}, and \ac{gdc} is effective.

The problem of over

Our examination leads us to the conclusion that regularization does not prove to be effective for graph-level tasks of machine learning.
The best results for graph-level prediction tasks are always achieved when we use no regularization.

\section{Future Work}
\label{sec:conclusion:future}

In our study, we found out that, indeed, for graph-level prediction tasks, regularization is not very helpful; we think that several interesting questions arise from there.

First, we looked at two well-known architectures which are commonly used.
However, we performed our experiments only on a limited number of datasets, leaving the possibility that results could differ on other datasets.

We introduced an interesting metric, which unfortunately fell out of scope in this research.
We encourage others to pick up where we left...



And there is the possibility, however small, that there are other types of regularization, that could perform well
Another proposed regularization technique
https://arxiv.org/pdf/2106.07971.pdf