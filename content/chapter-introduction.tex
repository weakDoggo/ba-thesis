% !TEX root = ../main.tex
% chktex-file 46
\chapter{Introduction}%
\label{sec:intro}

\pagenumbering{arabic}			% arabic page numbering
\setcounter{page}{1}			% set page counter

The field of \acs{ml} on graph-structured data has recently become an active topic of research.
One reason for this is the wide range of domains and problems that are expressible in terms of graphs.
Regularization techniques are commonly employed in node-level prediction tasks across diverse networks to mitigate over-smoothing and overfitting issues.
These methods perturb the values and introduce randomness, leading to improved results.
In graph-level tasks, however, the final output is a readout of aggregated nodes, hypothetically minimizing the impact of over-smoothing as the emphasis shifts from distinguishing individual nodes to capturing collective behavior.


\section{Motivation}%
\label{sec:intro:motivation}
% Mention GDC (including citation!)
Introducing controlled randomness might seem unnecessary for graph-level prediction tasks.
This shift in focus from individual nodes to aggregated outcomes challenges the conventional wisdom of applying regularization methods in graph-level prediction tasks.
Despite this counterintuitive aspect, our study aimed to empirically validate the effectiveness of regularization techniques in graph-level prediction tasks.
Our exploration of this uncharted territory was motivated by a desire to challenge existing assumptions and gain a nuanced understanding of the relationship between regularization methods and graph-level predictions.

\section{Research Questions}%
\label{sec:intro:questions}
In this thesis, we will answer the following research questions to assess the relevance of regularization for graph-level prediction tasks:
\begin{enumerate}
    \item Is regularization, specifically \ac{gdc}, effective in solving the problem of over-fitting and over-smoothing for graph-level prediction tasks?
    \item Is there a difference in performance between \ac{gcn} and \ac{gin} architectures regarding performance with regularization techniques?
    \item Are there similarities and differences between different regularization techniques in terms of performance?
\end{enumerate}
\section{Structure}%
\label{sec:intro:structure}

\paragraph{\Cref{sec:related}: \nameref{sec:related}}
We introduce a number of related studies.
% Problem Description + Implementation = implement 
\paragraph{\Cref{sec:implement}: \nameref{sec:implement}}
In this section, we focus on the implementation of regularization. Specifically, we explain, how using TensorFlows \texttt{gather} and \texttt{scatter} operations we implement all four regularization techniques by applying different random masks at different stages.
% eval contains discussion 
\paragraph{\Cref{sec:eval}: \nameref{sec:eval}}
Some text
\paragraph{\Cref{sec:conclusion}: \nameref{sec:conclusion}}
Finally, the results of this thesis are summarized and a brief outline of promising directions for future research is given.
