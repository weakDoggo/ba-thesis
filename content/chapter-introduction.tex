% !TEX root = ../my-thesis.tex
%
\chapter{Introduction}
\label{sec:intro}

\cleanchapterquote{You can’t do better design with a computer, but you can speed up your work enormously.}{Wim Crouwel}{(Graphic designer and typographer)}

\Blindtext[2][2]

\section{Postcards: My Address}
\label{sec:intro:address}

\textbf{Ricardo Langner} \\
Alfred-Schrapel-Str. 7 \\
01307 Dresden \\
Germany


\section{Motivation and Problem Statement}
\label{sec:intro:motivation}

There are a lot of things which have a graph structure.
Therefore many things can be represented as graphs.

Graph Neural Networks have been around and very successful in modeling various
graph related tasks.

There exist many different types of graph neural networks.
Such as GCN, GIN, GraphSAGE, GAN etc.

Graph nerral networks extract important information by aggregating neighborhood features.
Graph convolutional networks.

There are several Problems with this

\section{Results}
\label{sec:intro:results}


\subsection{Some References}
\label{sec:intro:results:refs}
\cite{WEB:GNU:GPL:2010,WEB:Miede:2011}

\section{Thesis Structure}
\label{sec:intro:structure}

\textbf{Chapter \ref{sec:related}} \\[0.2em]
\blindtext

\textbf{Chapter \ref{sec:system}} \\[0.2em]
\blindtext

\textbf{Chapter \ref{sec:concepts}} \\[0.2em]
\blindtext

\textbf{Chapter \ref{sec:concepts}} \\[0.2em]
\blindtext

\textbf{Chapter \ref{sec:conclusion}} \\[0.2em]
\blindtext
